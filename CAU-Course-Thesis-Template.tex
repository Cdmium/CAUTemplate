\documentclass[twoside,maketable]{cauthesis}
%%  oneside     用于单面打印(默认值)     
%%  twoside     用于双面打印,在标题页后会生成空白页,清除双数页。注:正规的课程论文应使用单面打印
%%  maketable   在摘要后打印目录
    \addbibresource{./bibsource.bib}%添加bib文件
    \title{中国农业大学课程论文\LaTeX{}模板}%输入标题
    \date{}%用于清除标题下的时间
    \author{}%用于清除标题下作者
    \tpitem{%请使用英文分号隔开,请严格按照格式填写
        论文题目;2;%输入2生成2行空白下划线
        课程名称;1;%输入1生成1行空白下划线
        任课老师;0;老师;%输入0并输入第三项自动填写内容
        班级;0;班级;
        学号;0;0123456789012;
        姓名;0;姓名;
        测试信息;0;测试过长信息会自动换行;%信息过长会自动换行
    }%无法输入过多信息,会引起错误,如有需求,可以手动修改cauthesis.cls中标记%1%处(修改信息与上方文字空白距离)或%2%处(信息行间距)
    \schoolyear{2018-2019}%输入学年信息,注释此命令为“20  -20  ”
    \semester{秋}%输入学期信息,注释此命令为“  ”
    \zhabstruct{
        这是使用\LaTeX{}编写的中国农业大学课程论文模板,参照中国农业大学课程论文格式编写(吐槽下,教务处的文件打不开),制作了可自定义的封面
        添加了Bib\LaTeX{}宏包作为文献引用,同时添加了许多常用宏包。

        编写过程主要参考胡振震编写的模板\cite{Huzhenzhen2018}。
    }%填写中文摘要
    \zhkeywords{\LaTeX{};中国农业大学;课程论文模板}%填写中文关键词,使用英文分号隔开
    \enabstruct{
        This the China Agricultural University Course Thesis \LaTeX{} Template.
        This template is made according to China Agricultural University course thesis writing format.
        Customizable title page is available.
        Bib\LaTeX{} marco is used to produce bibliography.

        The template made by HU Zhenzhen is used as the main reference in the process\cite{Huzhenzhen2018}.
    }%填写英文摘要,删除此命令可取消显示英文摘要和英文关键词
    \enkeywords{\LaTeX{};China Agricultural University;Thesis template}%填写英文关键词,使用英文分号隔开
\begin{document}
    \section{引言}
    \LaTeX{}是应该是世界上目前最专业的论文写作软件,有很完备的系统,以及数千的宏包支持,许多顶级期刊只接收\TeX{}格式的投稿,很多大学都有官方的\LaTeX{}模板。
    在中国\LaTeX{}的使用并不算普遍,为了方便之后课程论文的写作,同时推广\LaTeX{}的使用,编写了此模板。
    此模板可以解决课程论文的写作,并对一些复杂功能提供简便的使用。
    \section{声明}
    本模板根据中国农业大学课程论文格式制作,设计供中国农业大学学生使用。

    当前版本为2018/12/24 v1.2

    本模板是以article为基类,并加载ctex中文环境和xeCJKfntef中文字体。
    使用与article.cls基本相同,但必须保存为UTF8编码,建议使用\XeLaTeX{}+biber进行编译,建议的编译链为\XeLaTeX{}->biber->\XeLaTeX{}*2。

    模板加载了多种常用宏包,部分为实现模板所必须,其余可自行根据需要关闭或添加。

    模板发行包包括以下文件:

    \begin{table}[H]
        \begin{tabular}{ll}
            cauthesis.cls&模板主文件(必须)\\
            CAU-Course-Thesis-Template.tex&本文件源文件\\
            CAU-Course-Thesis-Template.pdf&本文件\\
            bibsource.bib&参考内容\\
            gb7714-2015.bbx&参考文献模板\\
            Logo.jpg&校徽\\
            中国农业大学课程论文写作格式.doc&编写模板的依据
        \end{tabular}
    \end{table}
    
    模板的最新版本可以通过\url{https://github.com/Cdmium/CAUTemplate}获取。

    有关模板的任何内容,如Bug提交、功能建议等均可通过GitHub或邮箱\url{yanyuxuan@cau.edu.cn}联系作者。

    版本更新记录\url{https://github.com/Cdmium/CAUTemplate/blob/master/Change%20Log.md}。

    本模板是LPPL协议下的项目,全文可在\url{https://www.latex-project.org/lppl.txt}查看。

    重要声明:
    \begin{level}{1}
        \item 任何个人和团体可以无限制的自由使用和更改此模板
        \item 本模板为非官方模板,模板作者对使用该模板所引起的后果不负任何责任
    \end{level}

    \section{使用}
    为了增加模板文件的兼容性,没有添加listing宏包,在本文中添加代码和说明,使用时请参照源文件使用。
    更多功能和方法请参考lshort、The TeX Book及宏包手册。

    \subsection{插入图片}
    图\ref{fig:1}%引用
    \begin{figure}[H]
        \centering
        \includegraphics{./Picture/Logo.jpg}
        \figcaption{演示插入图片}%建议使用定义好的\figcaption{}可以避免出现错误
        \label{fig:1}%引用标记
    \end{figure}

    \subsection{插入表格}
    表\ref{tab:1}%引用
    \begin{table}[H]
        \centering
        \begin{tabular}{lc}
            \hline
            Atom&Radius(nm)\\
            \hline
            Hydrogen&0.12\\
            Oxygen&0.14\\
            Nitrogen&0.15\\
            Carbon&0.17\\
            Sulfur&0.18\\
            Phosphorus&0.19\\
            \hline
        \end{tabular}
        \tabcaption{演示插入表格}%建议使用定义好的\tabcaption{}可以避免出现错误
        \label{tab:1}%引用标记
    \end{table}

    \subsection{参考文献}
    \cite{王夫之1845--}\cite{KENNEDY1975-339-360}%使用\cite{}引用bib文件中的内容
    \nocite{汪昂1881--}%使用\nocite{}引用但不会在文中显示[1],最后参考文献会显示
    %可以使用\nocite{*}一次引用所有内容

    \subsection{层次使用}
    定义了具有两个层次的带序号的罗列环境,基于enumerate环境,
    \begin{level}{1}
        \item 层次一标题
        \begin{level}{2}
            \item 层次二标题
            \item \ldots
        \end{level}
        \item \ldots
    \end{level}
    
    \section{宏包列表}
    \begin{center}
        \begin{supertabular}{lll}
            \hline
            宏包名称&功能&状态\\
            \hline
            geometry&控制页面布局&必须\\
            graghicx&插入图片&必须\\
            placeins&控制浮动体浮动范围&可选默认开启\\
            etoolbox&编程辅助&必须\\
            ifthen&编程辅助&必须\\
            ulem&多种字体修饰&必须\\
            array&表格拓展&必须\\
            \hline
            hyperref&超链接&必须\\
            titleref&章节标题超链接&必须\\
            fontspec&英文字体&必须\\
            ctex&中文环境&必须\\
            xeCJKfntef&中文字体&必须\\
            titlesec&章节标题格式&必须\\
            \hline
            bbding&特殊字体&可选默认开启\\
            pifont&特殊字体&可选默认开启\\
            xltxtra&特殊文本&可选默认开启\\
            mfloho&特殊文本&可选默认开启\\
            texames&特殊文本&可选默认开启\\
            \hline
            amsmath&数学&可选默认开启\\
            amssymb&数学&可选默认开启\\
            mathrsfs&数学字体&可选默认开启\\
            \hline
            xcolor&颜色控制&必须\\
            pgf&绘图&必须\\
            tikz&绘图&必须\\
            pgfplots&统计图&可选默认关闭\\
            pgftable&统计图表格&可选默认关闭\\
            \hline
            float&浮动体&必须\\
            multirow&表格合并行&可选默认开启\\
            booktabs&表格加强&可选默认开启\\
            longtable&长表格&可选默认关闭\\
            supertabular&跨页表格&可选默认开启\\
            tabularx&表格加强&必须\\
            subfigure&子图片&可选默认开启\\
            \hline
            ccpation&说明文字加强&必须\\
            url&网址&必须\\
            biblatex&参考文献&必须\\
            enumitem&罗列加强&必须\\
            \hline
        \end{supertabular}
    \end{center}
    
    \section{测试一级标题}
    \subsection{测试二级标题}
    \subsubsection{测试三级标题}

    \specialsection{致谢}%使用\specialsection{}生成没有编号但编入目录的章节。
    %请勿在此section下添加subsection或subsubsection会产生标号紊乱。
    \printbibliography%打印引用文献
\end{document}